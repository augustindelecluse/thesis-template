% For encoding
\usepackage[T1]{fontenc}
\usepackage[utf8]{inputenc}
\usepackage[english]{babel}


% theorem-like environments, see https://tex.stackexchange.com/questions/21227/example-environment for considerations
\usepackage{amsthm}
\theoremstyle{definition}
\newtheorem{example}{Example}[section]  % define a \begin{example} ... environment

% For quotes
\usepackage{epigraph}

% Additional math symbols
\usepackage{amssymb}

% Enable URL command with hyperlinks, while forcing correct line-breaks (see https://tex.stackexchange.com/a/3034)
% Also disable the neon boxes around hyperlinks (https://tex.stackexchange.com/a/12408)
% Allow URLs with correct line-breaking
% Configure hyperref to avoid duplicate anchors
\PassOptionsToPackage{hyphens}{url}
\usepackage[hidelinks,plainpages=false,pdfpagelabels,english,hypertexnames=false]{hyperref}

% Biblatex provides much more flexibility than bibtex for bibliography and citation management
% For quotes used by biblatex
\usepackage{csquotes}
% alphabetic style for labeled citations, maxbibnames to 99 to avoid "et al." in bibliography, giveninits to true to keep only given name initials
\usepackage[style=alphabetic,maxbibnames=99,giveninits=true,url=true]{biblatex}
% Add your bib files here
\addbibresource{biblio.bib}

% When calling "\fullcite", prevent it from "et al."ing
\preto\fullcite{\AtNextCite{\defcounter{maxnames}{99}}}
% Some bib fields might be useless to show, the following lines configure this
\AtEveryBibitem{% Clean up the bibtex rather than editing it
	\clearlist{language}
	\ifentrytype{online}{}{% Remove url except for @online
		\clearfield{url}
		\clearfield{urldate}
	}
}

% The following commands allow citing own contributions in the "Bibliographic notes" section without appearing in the "References" one
\DeclareBibliographyCategory{fullcited}
\newcommand{\mybibexclude}[1]{\addtocategory{fullcited}{#1}}
% The "\myfullcite" command to be used in the "Bibliographic notes" section
\newcommand{\myfullcite}[1]{\fullcite{#1}.\mybibexclude{#1}}
%\usepackage{cite}
%\usepackage{bibentry}
% Allow line breaks after periods in initials
\renewcommand{\bibinitperiod}{\adddot\addspace\allowbreak}
% Apply sloppy settings to the bibliography
\AtBeginBibliography{\sloppy}

% Correct line breaking in bib citations, must be loaded after biblatex
\usepackage{xurl}


% Linux Libertine is a nice font; notice that we scale the "\texttt", as the default tt font is too large
\usepackage[ttscale=.875]{libertine}
% Also adapt the math mode
\usepackage[libertine]{newtxmath}

% Enable to customize the format of the "\today" command
\usepackage{datetime}

% Add key-value interface for "\includegraphics" command
\usepackage{graphicx}

% Provide "\multirow" for tabular environment
\usepackage{multirow}

% To include PDF documents inside the current one
\usepackage{pdfpages}

% To tune the chapter format
\usepackage{pbox}
\usepackage[clearempty,newlinetospace,raggedright,small,explicit]{titlesec}

% Provide additional features to the verbatim environment
\usepackage{moreverb}

% Enable figure and table (sub)captioning, and configure their font
\usepackage[font=small]{caption}
% Defer caption setup for figures and tables until after \begin{document}
\AtBeginDocument{%
	% \captionsetup starred version gives no warning if no figure / table is present
	\captionsetup*[figure]{labelfont=bf,textfont=bf}
	\captionsetup*[table]{labelfont=bf,textfont=bf}
}
\usepackage[hypcap=false]{subcaption}

% Extend the interface of floating objects like figures and tables
\usepackage{float}

% Provide framed and coloured boxes
\usepackage{mdframed}

% Flexible handling of verbatim environment
\usepackage{fancyvrb}

% Macro package to create graphics, with its "user-friendly" syntax Tikz
\usepackage{pgf}
\usepackage{tikz}
% Enable drawing plots by directly specifying them in Tex
\usepackage{pgfplots}
\pgfplotsset{compat=1.18}
% This is just loading optimizations
\usetikzlibrary{positioning, fit, arrows.meta, shapes, arrows, decorations.markings, calc, decorations.pathreplacing}
\usepgfplotslibrary{groupplots}

% Provide facilities to insert (or not) spaces after macros (with "\xspace")
\usepackage{xspace}

% Provide the text companion font
\usepackage{textcomp}

% For figures in network protocol specifications
\usepackage{bytefield}

% For LuaTex, provide utility commands built-in in pdflatex
\usepackage{pdftexcmds}

% Provide support to rotate figures
\usepackage{rotating}

% Allow wrapping text around figures
\usepackage{wrapfig}

% Define colors
\usepackage{xcolor}

% Enable algorithm definition
\usepackage[linesnumbered, titlenumbered,ruled,noend]{algorithm2e}
\usepackage{amsmath}
%\usepackage{algorithm}
\usepackage[noend]{algpseudocode}

%\renewcommand{\algorithmicrequire}{\textbf{Input:}}
%\renewcommand{\algorithmicensure}{\textbf{Output:}}

% Change the section numbering depth to 3 levels
\setcounter{secnumdepth}{3}
% Idem for the table of contents
\setcounter{tocdepth}{3}

% Replace the item bullets by a nice gray square
\renewcommand{\labelitemi}{\color{gray} \scriptsize$\blacksquare$}

% Placing TODOs in Latex
\usepackage{todonotes}
% Proposed options: inline TODOs instead of margin ones, disable TODOs
%\presetkeys{todonotes}{inline,linecolor=red,backgroundcolor=red!25,disable}{}

% I hope you won't require this package once your thesis is completed :-)
\usepackage{lipsum}

% For the chapterformatting
\definecolor{gray65}{gray}{0.65}

\newcommand{\size}[2]{{\fontsize{#1}{0}\selectfont #2}}
\newcommand{\sizeline}[3]{{\fontsize{#1}{#2}\selectfont #3}}
\newenvironment{sizepar}[2]
{\par\fontsize{#1}{#2}\selectfont}
{\par}

\newcommand{\printchapternumber}{%
	\begingroup\renewcommand{\arraystretch}{0}%
	\begin{tabular}{@{}c@{}}\size{56}{\thechapter}\end{tabular}%
	\endgroup
}

\titleformat{\chapter}[block]
{\bfseries\filright}
{}
{0pt}
{%
	\parbox{.865\textwidth}{\raggedright\Huge\selectfont#1}%
	\hspace{6pt}%
	\color{gray65}\vrule width 1.5pt
	\hspace{6pt}%
	\printchapternumber
}

\titleformat{name=\chapter,numberless}[block]
{\bfseries\filright}
{}
{0pt}
{%
	\parbox{\textwidth}{\raggedright\Huge\selectfont#1}%
}

\titlespacing*{\chapter}{0pt}{0pt}{60pt}

%%%
% Short date format : May 2019
\newdateformat{monthyeardate}{\monthname[\THEMONTH] \THEYEAR}

% Clear an empty double page
\newcommand{\clearemptydoublepage}{%
	\newpage{\pagestyle{empty}\cleardoublepage}}